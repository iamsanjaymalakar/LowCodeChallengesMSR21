\section{Conclusions} \label{sec:conclusion}
LCSD is a new paradigm that enables the development of
software applications with minimal hand-coding using visual programming. We present an empirical study that provides insights about the types of topics low-code developers are discussing in Stack Overflow (SO). We find 13 low-code topics in our dataset of 4.6K SO posts (question + accepted answers). The posts are collected based on 19 SO tags belonging to the popular nine  LCSD platforms during our analysis. We categorize them into four high-level groups, namely Customization, Platform Adoption, Database, and Integration. Our findings reveal that developers find the external API Integration topic category the most challenging and Database category least difficult. Dynamic Event Handling is the most popular, as well as the most challenging topic. We find that there is a severe lack of good tutorial based documentation that deters smooth adaptation of LCSD. We hope that all of these findings will help various  LCSD stakeholders (e.g.,  LCSD platforms, practitioners, SE researchers) to take necessary actions to address the various  LCSD challenges. Since the growth indicates that this technology is likely to be widely adopted by various companies for their internal and customer-facing applications, platform providers should address the prevailing developers' challenges.


%We also manually examined SO discussions and classified the challenges practitioners face during different stages of agile software development life-cycle. We observed that ?, ? is the widely discussed topic in the community and Application development is the most difficult stage of for the practitioners. Many of the practitioners lack general programming and software engineering concepts.  We also presented how topics and challenges are distributed across LCDPs. We found that ? has lots of queries regarding deployment, platform ? has API integration related discussions. 

% Anindya - The above para can be added later if there is space after addressing the ? things